\documentclass[a4paper, 12pt]{article}

% Packages
\usepackage{graphicx}
\usepackage[utf8]{inputenc}
\usepackage[backend=biber]{biblatex}
\usepackage{ifxetex}

% TUB Font
\ifxetex
   % Muli font
   \usepackage{xltxtra}
   \setmainfont{Muli}
\else
   % Arial font
   \usepackage{times}
   \renewcommand{\rmdefault}{\sfdefault}
\fi
\linespread{1.2}

% Settings
\graphicspath{ {images/} }
\addbibresource{bibliography.bib}

% Variables
\newcommand{\thesistitle}{model2regex: Detecting DGAs with Regular Expressions Generated by a Language Model}
\newcommand{\thesisauthor}{Eric Schneider}
\newcommand{\matrno}{365800}
\newcommand{\supervisor}{Alexander \textsc{Warnecke}, Tammo \textsc{Kr\"uger}}

\begin{document}

\begin{titlepage}
	\centering
	\includegraphics[width=5cm]{tub-logo}\par\vspace{0.5cm}
	{Technische Universität Berlin \par}
	\vspace{2cm}
	{\large \textsc{Exposé}\par}
	\vspace{1cm}
	{\Large\bfseries \thesistitle\par}
	\vspace{2cm}
	%\vfill
	{\large \thesisauthor\par}
	{\large Matr. No. \matrno\par}
	\vspace{2cm}
	\includegraphics[width=5cm]{mlsec-logo-red2}\par\vspace{0.5cm}
	{Chair of Machine Learning and Security \par}
	{Prof. Dr. Konrad Rieck \par}
	\vfill
	supervised by\par
	\supervisor
	\vfill
	\today\par
\end{titlepage}

\section{Introduction}
%Some intro to the topic: What is it about? What is the specific problem that should be addressed in this work?
Domain Generating Algorithms (DGAs) are increasingly used in botnets as part of
command and control (C\&C) communication. Malware creators use these algorithms
to generate multiple possible domains each day and then have their malware
contact a small portion of them to obfuscate the real server they are getting
their instructions and updates from. This tactic gives the attacker a huge
advantage because protecting against means taking control of possible thousands
of domains while the attacker only needs to control a short lived domains
before the attack at the right time. Better protection may come from blocking
botnets at the source by recognizing communication with specific domains as
fraudulent or rather generated by a specific DGA family. DGAs however are
generated randomly and use different seeds from either specific dates, twitter
trends, hashes or word lists, therefore static blocklists may not be able to
keep up with blocking the communication at a network level. Deep Learning
approaches have shown great promises and are currently state of the art in
detecting Algorithmically-Generated Domains (AGD). 
However machine learned models can also be a black box and hide biases in the evaluation 
of these models. Also setting up these models in a filter of for example a firewall may
not be easily possible. However setting up a simple Regular Expression to filter out specific classes of AGDs
is the standard in many network protection solutions.
This thesis will therefore trying to attempt to use the power of machine learning to learn the structure
of domains generated by a DGA and use this information to then generate a regular expression (RegEx) 
which will match possible domains of that DGA. 


\section{Methodology}
%What methodology will be applied in this work? That is, what is the general
% strategy to solve the problem this work is concearned with?
The main methodology of this thesis will be applied research. Using currently established solutions
from the field of language processing. As data source of the learning process is a mix of real domains
from domain lists and self-generated domains using reverse engineered DGAs. \cite{baderj_dga}
The main focus of the research will be answering the two main questions. First, is it possible to
generate Regular Expressions in this setting? Second, do these regular expressions perform as well
as the current state of the art?


\section{Approach}
%How is the implementation of the strategy approached?
This project will need three phases. First the research phase, exploring the possibilities of
turning the internals of a language model into a regular expression. Second implementation of the
solution, 

\section{Evaluation}
%How is the degree of success of the applied method measured?
We will evaluate the result of this thesis by testing how well the generated regular expressions
will capture the learned structures of DGAs. We will compare how well these expressions will detect
AGDs. It is imperative that however the amount of false positives (benign domains detected as AGDs)
should be very low to not block valid connections with generated filter.
On the first

\section{Scope}
%What is the particular scope of your research? Which goals should be achieved?
%Which are optional and which are explicitly not part of the scope of this
%thesis?

\section{Related Work}
%Describe the state of the art of the field of research. Support your statements
%with appropriate sources \cite{fardin14}.

\clearpage

\printbibliography

\end{document}
